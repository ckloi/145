\documentclass[12pt,letterpaper]{article}
\begin{document}

\title{Creating the 'tsq' Package}
\date{March 21, 2017}
\author{Andrew Faust, Chi Kei Loi, Mano Wared, Tingting Zhu}
\maketitle

\section{Description}
The purpose of creating the ‘tsq’ package was to implement the Stack, Queue,
and Binary Tree data structures in R. However, creating these data structures
proved rather difficult, due to Rs lack of pointers. Fortunately, this obstacle
was avoided thanks to R’s vector operations. The data structures in this
package were all implemented using vectors and R6 reference classes.

\section{Stack and Queue}
When creating the Stack and Queue data structures, a lack of pointers is not a
huge issue. Both data structures can be thought of as arrays (or vectors), with
the last element being removed for Stacks, and the first element being removed
for Queues when popping. Therefore, that is exactly how these data structures
are implemented in this package. When being pushed to a Stack or a Queue, the
element is simply appended to the end of the vector. When popping, Stack
removes and returns the last element of the vector, while Queue removes and
returns the first. Space and time were not much of an issue, as the elements to
be popped are always at the front or back of the vector, and R's vector
operations allow these elements to be easily removed.

\section{Binary Tree}
Creating a Binary Tree without pointers proved much more of a challenge than
implementing Stack and Queue. Typically, Binary Trees are implemented by using
nodes that point to children nodes containing values lesser or greater than
their own. Using R’s matrix operations, however, allowed for a similar concept
to be implemented without the use of pointers.

The Binary Tree itself is represented using a single matrix. Each row of the
matrix represents a node in the tree, and the matrix has three columns: one for
the element within the node, and two for the row indices of the lesser and
greater children nodes. The tree initially consists of one row with all null
elements, with rows being appended whenever an element is pushed onto the tree.
When popping, the parent looks ahead to see if it’s left child had a left
child itself. If not, then the parent updates it’s left “pointer” to
bypass the left child, while leaving the child in the matrix.

While implementing the pop function, space and time efficiency had to be
considered. Would it be better to replace rows that have been popped, or to
simply leave them be as their parents would be updated to avoid them? The
former would prove to be much more space efficient, however time efficiency
would take a hit as the entire matrix would possibly have to be searched before
a popped row was found, if any. The latter, however, would be a problem for
space, as popped nodes are never truly removed from the tree. After careful
consideration, we chose the latter option, and thus our Binary Tree data
structure is time efficient, but not very efficient in terms of space.

\section{Why R6 Classes?}
When implementing these three data structures, we realized that the
specification of the pop function conflicts with the convention of the S3
class, and thus ran into a problem when using them. Since R is a functional
language, functions have no side effects, meaning that changing the data
structure’s object in the pop function only changes the copy that was passed
into the function, and not the object itself. Thus, the pop function needed to
return the modified object so that the actual object can be reassigned. By
doing so, we cannot return the popped value, which contradicts the
implementation of traditional pop method. As a result, we decided to use R6, a
powerful library that contains implementation of classes with reference
semantics. This made implementing the traditional pop function much easier as
we had direct access to the data structure, and could modify it and output the
popped value in the same function.

\section{Conclustion}
These classes are examples of the downsides of R, and how they can be overcome.
Using R's vector operations, we were able to mimic the use of pointers to
create Stack, Queue, and Binary Tree classes. We were also able to overcome the
limitations of the traditional S3 class by using the R6 reference class instead.

\end{document}
